\documentclass{article}

\usepackage{polski}
\usepackage[utf8]{inputenc}
\usepackage{graphicx}
\usepackage[hyphens]{url}
\usepackage{authblk}
\usepackage{tabularx}
\usepackage[a4paper, left=2cm, right=2cm, top=3cm, bottom=3cm]{geometry}
\usepackage{listings}
\usepackage{placeins}
\usepackage{hyperref}
\usepackage{biblatex}
\usepackage{amssymb}
\addbibresource{references.bib}

\usepackage[tablegrid,owncaptions]{vhistory}
\renewcommand{\vhhistoryname}{Historia zmian}
\renewcommand{\vhversionname}{Wersja}
\renewcommand{\vhdatename}   {Data}
\renewcommand{\vhauthorname} {Autor}
\renewcommand{\vhchangename} {Opis zmian}

\renewcommand\figurename{Rys.}
\renewcommand\lstlistingname{Wydruk}
%

%------------------------------------------------------------------------
\title{{\Huge  Projekt Systemy Cyfrowe}\\ {\Large Zespół nr 1} }

\author{Dominik Grzybowski \and  Krzysztof Zdulski \and Michał Wawrzyńczak \and Paweł Gryka \and Bartek Walusiak}

\affil{Politechnika Warszawska, Wydział Elektroniki i Technik Informacyjnych}
\date{\today}

%------------------------------------------------------------------------
\begin{document}
	
\maketitle
%------------------------------------------------------------------------
% Historia zmian
\begin{versionhistory}
	\vhEntry{1.0}{13.03.2020}{DG|KZ|MW|PG|BW}{Pierwsza wersja raportu Etapu 1}
\end{versionhistory}

%------------------------------------------------------------------------

\tableofcontents

%------------------------------------------------------------------------
\section{Wstęp}
\label{sec:wstep}


Otrzymano wiadomość od tajnego agenta, z której wynika, że zdobył on ważne informacje. Niestety, wiadomość jest uszkodzona i najprawdopodobniej niepełna.

\begin{quote}
\texttt{I gained important information. For lack of time and possibilities, I encoded pdf files using tea exactly according to what Wiki says. I send the package by a trusted courier, so it will take time. In Caesar's way, the password for the cipher is Narq oy zn ...... }
\end{quote}

Zgodnie z tą wiadomością materiały w postaci zakodowanych plików w formacie 
PDF zostały zakodowane prawdopodobnie prostym algorytmem i wysłane bezpiecznym
kanałem. Powinny one wkrótce dotrzeć.
Klucz do szyfru jest niepełny i wygląda na zakodowany.\\
 
Powierzone nam zadanie, to zaprojektowanie i zrealizowanie układu FPGA, które umożliwi odszyfrowanie przesłanych plików


\section{Organizacja prac}
\label{sec:organizacja}



\subsection{Design Thinking}
\label{sec:design_thinking}

\subsubsection{Empathize}
\label{sec:empathize}

Żeby jak najlepiej zrozumieć potrzeby użytkownika jeden z członków zespołu
postawił się w jego roli, a reszta zadawała pytania 
co do jego oczekiwań w stosunku do planowanego rozwiązania. 

\subsubsection{Define}
\label{sec:define}

Na podstawie przeprowadzonego wywiadu sformułowaliśmy listę potrzeb użytkownika:

\begin{itemize}
    \item Szybkość działania systemu jest kluczowa - HYDRA atakuje :o
    \item Użytkownikowi zależy na szybkim współdzieleniu zdekodowanego pliku z telefonem
    \item Nieważna jest mobilność rozwiązania - odbierany plik będzie przetwarzany w bazie S.H.I.E.L.D.
    \item Nie ma potrzeby szyfrowania kanału transmisji - sam plik jest już na wstępie zaszyfrowany
    \item Łatwość w użytkowaniu (GUI)
\end{itemize}

\subsubsection{Ideate}
\label{sec:ideate}

Zgodnie z założeniami tego punktu, wygenerowaliśmy wiele pomysłów:

\begin{itemize}
    \item Maszyna deszyfrująca znajduje się w satelicie i komunikujemy się z nią poprzez teleportację elektronów
    \item Jak wyżej ale komunikacja laserem
    \item Program w C + web interface
    \item Wynająć 100 zdolnych matematyków, którzy będą odszyfrowywali wiadomości na kartce
    \item Wytresować stado małp, które będą odszyfrowywały wiadomości
    \item Realizacja sprzętowa na FPGA + interface w Pythonie
    \item Poproszenie agenta, żeby nie szyfrował przesyłanych wiadomości (najszybsza opcja)
\end{itemize}

Analizując wygenerowane pomysły, stwierdziliśmy, że najlepszym podejściem będzie realizacja na FPGA.

\subsubsection{Prototype}
\label{sec:prototype}

TODO

\subsubsection{Test}
\label{sec:test}

TODO

\subsection{Zarządzanie projektem}
\label{sec:zarządzanie_projektem}

\textit{Zarządzam ja} $\sim$ Krzysztof Zdulski 2020

\subsubsection{Metody}
\label{sec:narzędzia}
Twarda dyscyplina kurcze blaszuszka ma być

\subsubsection{Narzędzia}
\label{sec:narzędzia}
Za nasz podstawowy zestaw narzędzi, którymi będziemy posługiwać się w trakcie realizacji projektu wybraliśmy:
\begin{itemize}
    \item Microsoft Teams - Zarządzanie plikami i dostarczanie sprawozdań do koordynatora
    \item Overleaf - Edytor plików LaTeX online
    \item Discord - Grupowa komunikacja głosowa
    \item Quartus Prime - IDE na plażę
    \item GitHub - słabszy odpowiednik \href{https://guthub.net/}{GutHub}
\end{itemize}
%------------------------------------------------------------------------
\section{Informacje podstawowe}
\label{sec:informacje_podstawowe}

TODO

%------------------------------------------------------------------------
\section{Koncepcja}
\label{sec:koncepcja}

TODO

%------------------------------------------------------------------------
\section{Implementacja}
\label{sec:implementacja}

TODO

\section{Uruchomienie}
\label{sec:uruchomienie}

TODO 
	
\section{Podsumowanie}
\label{sec:podsumowanie}

TODO


\printbibliography
	
\end{document}
